\documentclass[conference,compsoc,onecolumn]{IEEEtran}
\usepackage{./macros}

\begin{document}
\vspace{3cm}
\title{CS5300 Theory Assignment 3}
\author{Gautam Singh\\CS21BTECH11018}
\maketitle

\bigskip

\begin{enumerate}
    \item Suppose thread \(A\) acquires and releases this lock. Here, thread
    \(A\)'s own node is referenced by both \texttt{tail} and \texttt{myNode}.
    Now, when \(A\) tries to acquire the lock once again, the following happens.
    \begin{enumerate}
        \item Thread \(A\) sets it's \texttt{locked} field to \texttt{true} in
        line 8.
        \item Thread \(A\) swaps \texttt{tail} with \texttt{qnode} in line 10,
        but obtains its thread-local node again. In other words, \(A\) becomes
        its own predecessor.
        \item Thread \(A\) then spins on \texttt{pred.locked} which corresponds
        to the \texttt{locked} field in its own memory. In other words, thread
        \(A\) deadlocks itself.
    \end{enumerate}

    On the other hand, the MCS lock sets the tail node of the queue to
    \texttt{null} if there are no further threads waiting in the queue to
    signify the queue is empty. In this way, the thread reclaims its memory and
    does not deadlock itself.

    \item Here are the implementations of \texttt{isLocked()} for various types
    of locks.
    \begin{enumerate}
        \item A test-and-set lock is acquired if the atomic boolean variable
        \texttt{state} is set, which is shown in \autoref{code:tas-islocked}.

        \item A CLH lock is acquired when the \texttt{locked} field of the tail
        of the queue is set. Thus, we can simply check that to infer whether the
        lock is acquired. This is shown in \autoref{code:clh-islocked}.

        \item An MCS lock is acquired when the \texttt{tail} of the queue is set
        and its \texttt{locked} field is also set. Thus, we obtain a reference
        to the tail and check these two conditions. This is depicted in
        \autoref{code:mcs-islocked}.
    \end{enumerate}

    \inputminted[breaklines,linenos,breaksymbolleft=]{java}{codes/TASLock.java}
    \captionof{listing}{Implementation of \texttt{isLocked} for \texttt{TASLock}. \label{code:tas-islocked}}

    \inputminted[breaklines,linenos,breaksymbolleft=]{java}{codes/CLHLock.java}
    \captionof{listing}{Implementation of \texttt{isLocked} for \texttt{CLHLock}. \label{code:clh-islocked}}

    \inputminted[breaklines,linenos,breaksymbolleft=]{java}{codes/MCSLock.java}
    \captionof{listing}{Implementation of \texttt{isLocked} for \texttt{MCSLock}. \label{code:mcs-islocked}}

    \item The implementation of a "nested" readers-writers lock is shown in
    \autoref{code:nested}.
    \inputminted[breaklines,linenos,breaksymbolleft=]{java}{codes/NestedReadWriteLock.java}
    \captionof{listing}{Implementation of nested readers-writers lock.\label{code:nested}}

    To argue correctness of the lock, observe the following.

    \begin{enumerate}
        \item Suppose a thread holds the write lock. Then, it must have acquired
        the read lock as well and was the only reader in the system. Another
        thread looking to acquire the read lock must wait on the condition in
        line 46 of \autoref{code:nested} since there is a writer thread. Since
        it cannot acquire the read lock, it definitely cannot acquire the write
        lock. Thus, the writer thread has exclusive write access.

        \item Suppose a thread holds the read lock. Then, any other thread
        looking to acquire the write lock will first acquire the read lock, but
        will then wait on the condition in line 46 of \autoref{code:nested}
        since there are at least two reader threads.
    \end{enumerate}

    This implementation is fair for the writers due to the FIFO re-entrant lock
    used in acquiring the read and write locks. All writers will eventually
    acquire the write lock.
\end{enumerate}
\end{document}