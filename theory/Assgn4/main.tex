\documentclass[conference,compsoc,twocolumn]{IEEEtran}
\usepackage{./macros}

\begin{document}
\vspace{3cm}
\title{CS5300 Theory Assignment 4}
\author{Gautam Singh\\CS21BTECH11018}
\maketitle

\bigskip

\begin{enumerate}
    \item Consider the cases of the following method calls overlapping with the
    \texttt{add} method.
    \begin{enumerate}
        \item \texttt{add}: If the other thread is inserting a new element at
        another position, then the threads will lock different \texttt{pred}
        nodes and there will not be any issues in validations and insertions. If
        both threads are inserting at the same position, then one of them will
        acquire the lock on \texttt{pred} and insert their new element. The
        other thread will acquire the lock but fail validation, following which
        it will attempt to validate and attempt to insert its new element again
        from the start of the \texttt{while} loop in line 3.

        \item \texttt{remove}: Suppose the \texttt{curr} of the \texttt{add}
        method is being removed. Then, the threads will have to lock the same
        \texttt{pred} node and hence the operations will proceed sequentially,
        with the latter operation failing validation. If the \texttt{pred} of
        the \texttt{add} call is being removed, then a similar argument follows
        since this \texttt{pred} node must be locked by both calls.

        \item \texttt{contains}: This will not cause any issues and is
        linearizable with the concurrent \texttt{add}, even if the element being
        searched for is concurrently being added.
    \end{enumerate}

    Thus, in all cases, we need to only lock the \texttt{pred} node during
    insertion.

    \item Setting the \texttt{next} field of a node to be deleted to
    \texttt{null} would effectively de-link the list, destroying information
    about the next nodes that may not be deleted. Other concurrent calls can
    travel through marked nodes in the \texttt{LazyList} implementation,
    especially in the interval between logical deletion (by marking the node)
    and physical deletion (actually swinging the \texttt{next} pointer). Hence,
    this change will break the lazy implementation.

    This change will not work even in case of the \texttt{LockFreeList} since
    the \texttt{find} method performs the physical deletions of marked nodes,
    and hence it travels through logically deleted nodes as well.

    \item
    \begin{enumerate}
        \item The \texttt{compareAndSetMark} method does not check if
        \texttt{curr.next} is equal to \texttt{succ} unlike the original
        algorithm. This makes the list implementation incorrect. Between lines
        54 and the (modified) line 55, a concurrent addition might have taken
        place changing the successor node of \texttt{curr}, thus
        \texttt{pred.next} will be incorrectly set in line 58, skipping the new
        nodes inserted after \texttt{curr}.

        \item The \texttt{attemptMark} method is atmoic and can spuriously fail.
        It checks if \texttt{cur.next} equals \texttt{succ} from line 54 and
        then only marks the node for deletion. This maintains the correctness of
        the list.
    \end{enumerate}

    \item The minimum number of blocks to be used is
    \(\left\lceil\frac{8000}{1024}\right\rceil = 8\). This means that there are
    \(8 \times 1024 = 8192\) threads in the grid. The programmer should
    configure the kernel call to have 8 blocks and 1024 threads per block. In
    CUDA, this would be \texttt{kernel<<< ceil(8000/1024.0), 1024 >>> (args)} or
    \texttt{kernel<<< 8, 1024 >>> (args)}.
\end{enumerate}

\end{document}