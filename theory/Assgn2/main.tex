\documentclass[conference,compsoc]{IEEEtran}
\usepackage{./macros}

\begin{document}
\vspace{3cm}
\title{CS5300 Theory Assignment 2}
\author{Gautam Singh\\CS21BTECH11018}
\maketitle

\bigskip

\begin{enumerate}
    \item \textbf{True.} If thread \(A\)'s \texttt{read()} call does not overlap
    any \texttt{write()} call to its location, then this call will return the
    most recently written value into \texttt{s\_table[\(A\)]}. If it does, then
    it will return the new value or the old value since the component registers
    are regular, which is acceptable. Thus, the construction yields a regular
    Boolean MRSW register.

    \item \textbf{True.} Since the SRSW registers are regular, no read call can
    return a value from either the future or the distant past. We only need to
    check if it is possible that \(R^i \rightarrow R^j\) and \(R^i\) returns a
    value later than the value returned by \(R^j\). Since the SRSW registers are
    regular, this can only happen when both reads overlap a single write.
    
    Suppose \(R^i\) returns the old value. Then, \(R^j\) can return either the
    old or new value, depending on whether the diagonal entry for that thread
    has been updated by the writer thread, which is acceptable. 
    
    Otherwise, if \(R^i\) returned the new value, it would have also updated all
    the values in its column to reflect the newly read value. Then, \(R^j\)
    would scan its row to find this updated value written by \(R^i\) (if not
    written by the writer thread) and thus return the same value as it has the
    highest timestamp.

    Hence, the given construction still remains atomic even if SRSW regular
    registers are used.

    \item Obviously, the reader thread cannot read values from the future since
    the entry in \texttt{r\_bit} will not be set. Suppose \(W^i \rightarrow
    W^j\), with values \(v_i\) and \(v_j\) being written. If \(v_j \le v_i\),
    then a read following \(W^j\) will return \(v_j\). If \(v_j > v_i\), then
    \texttt{r\_bit[\(v_j\)]} is set and \texttt{r\_bit[\(v_i\)]} is unset, so a
    read after \(W^j\) will return \(v_j\).
    
    Suppose \(R^i \rightarrow R^j\), and \(R^i\) returns \(v_i\). Then, the
    first index with a set bit in \texttt{r\_bit} is \(v_i\). If no writes
    overlap with both these reads, then \(R^j\) will return \(v_i\) or a newly
    written value whose write overlaps only with \(R^j\). Now, suppose a write
    \(W\) overlaps with both reads and writes a value \(v_j\). If \(R^i\)
    returns \(v_j\), it must mean that \(v_j\) is the smallest index with a set
    bit in \texttt{r\_bit}. Hence, \(R^j\) will also return \(v_j\).

    This analysis shows that the given construction is an atmoic MRSW register.

    \item Clearly, this construction guarantees a safe register, since the
    returned value is also a 64-bit value. However, if a read operation overlaps
    with a write operation, then it can read the new upper 32 bits and the old
    lower 32 bits. This is characteristic of neither regular nor atomic
    registers. Thus, the strongest property satisfied by this construction is
    that of a safe register.
\end{enumerate}
\end{document}